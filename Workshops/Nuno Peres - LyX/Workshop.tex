%% LyX 2.3.6 created this file.  For more info, see http://www.lyx.org/.
%% Do not edit unless you really know what you are doing.
\documentclass[portuges]{article}
\usepackage[T1]{fontenc}
\usepackage[latin9]{inputenc}
\usepackage{bm}
\usepackage{graphicx}
\usepackage{esint}

\makeatletter

%%%%%%%%%%%%%%%%%%%%%%%%%%%%%% LyX specific LaTeX commands.
%% Because html converters don't know tabularnewline
\providecommand{\tabularnewline}{\\}
%% A simple dot to overcome graphicx limitations
\newcommand{\lyxdot}{.}


%%%%%%%%%%%%%%%%%%%%%%%%%%%%%% Textclass specific LaTeX commands.
\newcommand{\lyxaddress}[1]{
	\par {\raggedright #1
	\vspace{1.4em}
	\noindent\par}
}

\makeatother

\usepackage{babel}
\begin{document}
\title{Workshop on LyX}
\author{N. M. R. Peres}
\maketitle

\lyxaddress{Univeristy of Minho, Braga, Portugal}
\begin{abstract}
Nesta workshop iremos ver o funcionamento b�sico do LyX, incluindo
o uso de floats.
\end{abstract}
\tableofcontents{}

\section{Basic notions}
\begin{enumerate}
\item Equations
\begin{enumerate}
\item Formulas
\item Inline equations
\end{enumerate}
\item Latex commands
\item Flots
\item Bibliography
\end{enumerate}
These four topics will be covered in this workshop.

\subsection{Integrals}

My fisrt equation
\begin{equation}
n!=n(n-1)(m-2)\cdots2\times1\label{eq:rede}
\end{equation}


\subsection{Equations using LaTeX}

\begin{equation}
\int_{0}^{2}e^{-x}dx=-e^{-x}\vert_{0}^{2}=1-e^{-2}.
\end{equation}

Consideremos que a for�a � da forma

\begin{equation}
f(x)=\prod_{i=1}^{n}\frac{(-1)^{i}}{x^{i}},
\end{equation}

ent�o a segunda lei de Newton � dada por $m\ddot{x}=f(x)$

\begin{equation}
\int_{0}^{1}e^{-x}dx=\left.-e^{-x}\right|_{0}^{1}=1-e^{-1}\label{eq:borboleta}
\end{equation}

Equation (\ref{eq:borboleta}). This equation was first found in ref.
\cite{Buterfly}

\section{Tables}

\begin{table}
\begin{centering}
\begin{tabular}{|c|c|c|}
\hline 
$x$ & $y$ & $z$\tabularnewline
\hline 
\hline 
1 & 2 & 3\tabularnewline
\hline 
4 & 5 & 6\tabularnewline
\hline 
\end{tabular}
\par\end{centering}
\caption{Position of the bike in two different experiments\label{tab:Position-of-the-Newton-second-law}.}
\end{table}

\begin{equation}
\hat{L}_{z}=\left[\begin{array}{ccc}
1 & 0 & 0\\
0 & 0 & 0\\
0 & 0 & -1
\end{array}\right].\label{eq:Lz_angular_momentum}
\end{equation}
The $z-$component of the spin 1 angular momentum is given by Eq.
(\ref{eq:Lz_angular_momentum}). For spin 1/2 the vector of the Pauli
matrices is
\begin{equation}
\overrightarrow{\sigma}=(\sigma_{x},\sigma_{y},\sigma_{z}).
\end{equation}
The Hamiltonian of a spin 1/2 in a magnetic field reads
\begin{equation}
H=g\overrightarrow{B}\cdot\overrightarrow{\sigma}.
\end{equation}
Writing the Hamiltonian in bold math
\begin{equation}
H=g\mathbf{B}\cdot\bm{\sigma}.
\end{equation}
The $\sigma_{y}$ Pauli matrix is defined as
\begin{equation}
\sigma_{y}=\left(\begin{array}{cc}
0 & -i\\
i & 0
\end{array}\right).
\end{equation}


\section{Flots (Figures)}

\begin{figure}[t]
\begin{centering}
\includegraphics[width=5cm]{\string"WhatsApp Image 2022-12-15 at 19.26.26\string".jpeg}
\par\end{centering}
\caption{My dream bi\label{fig:My-dream-bike.}ke.}

\end{figure}

Figure \ref{fig:My-dream-bike.} was the bike used in the experiments
whose data is in table \ref{tab:Position-of-the-Newton-second-law}.
The characteristic of the bike can be found in ref. \cite{Bike_dic}
\begin{thebibliography}{1}
\bibitem{Bike_dic} N. M. R. Peres, \emph{Dictionary of mountain bike},
(Oxford University, Oxford, 2023).

\bibitem{Buterfly} N. M. R. Peres, Ensaio sobre borboletas, (Cambridge
University Press, Cambridge, 2055).
\end{thebibliography}

\end{document}
